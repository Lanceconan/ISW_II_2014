
\section{Algunos Framework}

\subsection{Codeigniter}

Algunos de los puntos más interesantes sobre este framework, sobre todo en comparación con otros productos similares, son los siguientes:


\begin{itemize}

\item Versatilidad: Quizás la característica principal de CodeIgniter, en comparación con otros frameworks PHP. CodeIgniter es capaz de trabajar la mayoría de los entornos o servidores, incluso en sistemas de alojamiento compartido, donde sólo tenemos un acceso por FTP para enviar los archivos al servidor y donde no tenemos acceso a su configuración.


\item Compatibilidad: CodeIgniter es compatible con la versión PHP 4, lo que hace que se pueda utilizar en cualquier servidor, incluso en algunos antiguos. Por supuesto, funciona correctamente también en PHP 5.

\item Actualizado: Desde la versión 2 de CodeIgniter ya solo es compatible con la versión 5 de PHP. Para los que todavía usen PHP 4 pueden descargar una versión antigua del framework, como CodeIgniter V 1.7.3, que todavía era compatible. Estas versiones están en la página de descargas de CodeIgniter.

\item Facilidad de instalación: No es necesario más que una cuenta de FTP para subir CodeIgniter al servidor y su configuración se realiza con apenas la edición de un archivo, donde debemos escribir cosas como el acceso a la base de datos. Durante la configuración no necesitaremos acceso a herramientas como la línea de comandos, que no suelen estar disponibles en todos los alojamientos.

\item Flexibilidad: CodeIgniter es bastante menos rígido que otros frameworks. Define una manera de trabajar específica, pero en muchos de los casos podemos seguirla o no y sus reglas de codificación muchas veces nos las podemos saltar para trabajar como más a gusto encontremos. Algunos módulos como el uso de plantillas son totalmente opcionales. Esto ayuda muchas veces también a que la curva de aprendizaje sea más sencilla al principio.

\item Ligereza: El núcleo de CodeIgniter es bastante ligero, lo que permite que el servidor no se sobrecargue interpretando o ejecutando grandes porciones de código. La mayoría de los módulos o clases que ofrece se pueden cargar de manera opcional, sólo cuando se van a utilizar realmente.

\item Documentación tutorializada: La documentación de CodeIgniter es fácil de seguir y de asimilar, porque está escrita en modo de tutorial. Esto no facilita mucho la referencia rápida, cuando ya sabemos acerca del framework y queremos consultar sobre una función o un método en concreto, pero para iniciarnos sin duda se agradece mucho.

\end{itemize}

 \newpage

\subsection{Zend framework}

Entre las principales características que podemos mencionar, se encuentran:

\begin{itemize}

\item Basado en PHP
\item Esta orientado a objetos
\item Usa el paradigma MVC, aunque no al 100%
\item Es open source
\item Ofrece un gran rendimiento
\item Ofrece una capa de abstracción a bases de datos fácil de usar
\item Facilita el acceso a los servicios web de distintas compañías como Google o Microsoft
\item Cuenta con una gran comunidad de desarrolladores que contribuyen con el mantenimiento y mejora del proyecto

\end{itemize}

Por otra parte, Zend Framework esta creado a partir de una infinidad de librerías y clases. Lo que lo hace una espada de doble filo; por un lado, es sencillo para los desarrolladores utilizar sólo las librerías que se necesitan; pero por otra parte, configurar el framework por primera vez no esta tan fácil ya que estamos ante un montón de clases juntas. Es por ello que Zend cuenta con una herramienta para crear y estructurar nuestra aplicación, herramienta que vamos a utilizar más adelante.

 \newpage

\subsection{CakePHP}

CakePHP es un marco de desarrollo [framework] rápido para PHP, libre, de código abierto. Se trata de una estructura que sirve de base a los programadores para que éstos puedan crear aplicaciones Web. El principal objetivo es que se pueda trabajar de forma estructurada y rápida, sin pérdida de flexibilidad.

Con CakePHP el desarrollo web ya no es monótono porque ofrece las herramientas para que se empiece a escribir el código que realmente  se necesita: la lógica específica de lu aplicación. 

CakePHP tiene un equipo de desarrolladores y una comunidad activos, lo que añade valor al proyecto. Con CakePHP, además de no tener que reinventar la rueda, el núcleo de tu aplicación se mejora constantemente y está bien probado.

Esta es una lista breve con las características de las que disfrutarás al utilizar CakePHP:

\begin{itemize}

\item Comunidad activa y amistosa
\item Licencia flexible
\item Compatible con PHP4 y PHP5
\item CRUD integrado para la interacción con la base de datos
\item Soporte de aplicación [scaffolding]
\item Generación de código
\item Arquitectura Modelo Vista Controlador (MVC)
\item Despachador de peticiones [dispatcher], con URLs y rutas personalizadas y limpias
\item Validación integrada
\item Plantillas rápidas y flexibles (sintaxis de PHP, con ayudantes[helpers])
\item Ayudantes para AJAX, Javascript, formularios HTML y más
\item Componentes de Email, Cookie, Seguridad, Sesión y Manejo de solicitudes
\item Listas de control de acceso flexibles
\item Limpieza de datos
\item Caché flexible
\item Localización
\item Funciona en cualquier subdirectorio del sitio web, con poca o ninguna configuración de Apache

\end{itemize}

 \newpage

\subsection{Yii frameworkk}

El proyecto Yii comenzó el 1 de enero de 2008, con el fin de solucionar algunos problemas con el framework PRADO. Por ejemplo, PRADO es lento manejando páginas complejas, tiene una curva de aprendizaje muy pronunciada y tiene varios controles que dificultan la personalización, mientras que Yii es mucho más fácil y eficiente.5 En octubre de 2008 después de 10 meses de trabajo en privado, la primera version alfa de Yii fue lanzada. El 3 de Diciembre del mismo año, Yii 1.0 fue formalmente presentado.4

Algunas características de Yii incluyen:

\begin{itemize}

\item Patrón de diseño Modelo Vista Controlador (MVC).
\item Database Access Objects (DAO), query builder, Active Record y migración de base de datos.
\item Integración con jQuery.
\item Entradas de Formulario y validacion.
\item Widgets de Ajax, como autocompletado de campos de texto y demás.
\item Soporte de Autenticación incorporado. Además soporta autorización via role-based access control (RBAC) jerarquico.
\item Personalización de aspectos y temas.
\item Generación compleja automática de WSDL, especificaciones y administración de peticiones Web service.
\item Internacionalización y localización (I18N and L10N). Soporta traducciones, formato de fecha y hora, formato de números, y localización de la vista.
\item Esquema de caching por capas. Soporta el cache de datos, cache de páginas, cache por fragmentos y contenido dinámico. El medio de almacenamiento del cache puede ser cambiado.
\item El manejo de errores y logging. Los errores son manejados y personalizados, y los log de mensajes pueden ser categorizados, filtrados y movidos a diferentes destinos.
\item Las medidas de seguridad incluyen la prevención cross-site scripting (XSS), prevención cross-site request forgery (CSRF), prevención de la manipulación de cookies, etc.
\item Herramientas para pruebas unitarias y funcionales basados en PHPUnit y Selenium.
\item Generación automática de codigo para el esqueleto de la aplicación, aplicaciones CRUD, etc.
\item Generación de codigo por componentes de Yii y la herramienta por linea de comandos cumple con los estándares de XHTML.
\item Cuidadosamente diseñado para trabajar bien con código de terceros. Por ejemplo, es posible usar el código de PHP o Zend Framework en una aplicación Yii.

\end{itemize}

 \newpage

\subsection{Symphony}

Symfony es un completo framework diseñado para optimizar, gracias a sus características, el desarrollo de las aplicaciones web. Para empezar, separa la lógica de negocio, la lógica de servidor y la presentación de la aplicación web. Proporciona varias herramientas y clases encaminadas a reducir el tiempo de desarrollo de una aplicación web compleja. Además, automatiza las tareas más comunes, permitiendo al desarrollador dedicarse por completo a los aspectos específicos de cada aplicación. El resultado de todas estas ventajas es que no se debe reinventar la rueda cada vez que se crea una nueva aplicación web.

Symfony está desarrollado completamente con PHP 5. Ha sido probado en numerosos proyectos reales y se utiliza en sitios web de comercio electrónico de primer nivel. Symfony es compatible con la mayoría de gestores de bases de datos, como MySQL, PostgreSQL, Oracle y SQL Server de Microsoft. Se puede ejecutar tanto en plataformas *nix (Unix, Linux, etc.) como en plataformas Windows. A continuación se muestran algunas de sus características.

\begin{itemize}

\item Su código, y el de todos los componentes y librerías que incluye, se publican bajo la licencia MIT de software libre.
\item La documentación del proyecto también es libre e incluye varios libros y decenas de tutoriales específicos.
\item Aprender a programar con Symfony te permite acceder a una gran variedad de proyectos: el framework Symfony2 para crear aplicaciones complejas, el micro framework Silex para sitios web sencillos y los componentes Symfony para otras aplicaciones PHP.
\item Según GitHub, Symfony es el proyecto PHP más activo, lo que garantiza que nunca te quedarás atrapado en un proyecto sin actividad. Además, el líder del proyecto, Fabien Potencier, es la segunda persona más activa del mundo en GitHub (ver datos).
\item Aunque en su desarrollo participan cientos de programadores de todo el mundo, las decisiones técnicas importantes siempre las toma Fabien Potencier, líder del proyecto. Esto evita el peligro de que surjan forks absurdos y la comunidad se fragmente.
\item Los componentes de Symfony son tan útiles y están tan probados, que proyectos tan gigantescos como Drupal 8 están construidos con ellos.
\item En todo el mundo se celebran varias conferencias dedicadas exclusivamente a Symfony. Para que te hagas una idea del tamaño de la comunidad, la conferencia Symfony española (llamada deSymfony) es el evento PHP más grande del país.
 
\end{itemize}

 \newpage

\subsection{Laravel}

Laravel es un reciente framework de PHP que es fácil de aprender y ofrece una muy interesante propuesta, sobre todo para los desarrolladores que anden en busca de una herramienta eficiente y de rápido aprendizaje, en el desarrollo de proyectos web.
Laravel se inició el año 2011, y aprovecha las mejoras de PHP 5.3 ofreciendo una sintáxis clara y simple en la creación de código PHP. Se pueden escribir aplicaciones web con muy pocas líneas de código que además fáciles de entender, incluso para un programador recién iniciado.
Laravel, debido a su corta curva de aprendizaje, similar a Codeigniter, pero con mucho mejor diseño, se está haciendo de una creciente popularidad en la comunidad de desarrolladores PHP, y eso da confianza en migrar a este framework.

Las principales característas que este moderno Framework PHP ofrece son:


\begin{itemize}

\item Una completa y concisa documentación que es muy sencilla de leer y comprender. Con código de ejemplo que es elegante y expresivo, facilitando significativamente aprendizaje del framework, incluso sólo observando el código.
\item Un ORM para manejar la capa de persistencia de datos de manera muy simple, con sólo un par de líneas de código se puede hacer mucho. Además, maneja con efectividad las distintas relaciones entre las tablas de una base de datos.
\item Un poderoso administrador de extensiones (Bundles), en el cual podemos instalarlo inmediatamente, algunos valiosos Bundles ya están disponible en la propia página de Laravel: http://bundles.laravel.com/
\item Es un proyecto Open Source con licencia MIT, de uso libre.
\item Las extensiones o Bundles, no sólo nos ayudan a incorporar nuevos módulo en nuestra aplicación, sino que además nos proveen una ruta para modularizar nuestras propias aplicaciones web, un aspecto muy valioso en el desarrollo web profesional.
\item Laravel se encuentra en su versión 4. 
\item Laravel dispone de un amplio material educativo, más allá de la documentación oficial, para aprender a usar el framework con rapidez. Se podría decir que leyendo su documentación en un par de horas se puede comenzar a programa en Laravel, es todo muy rápido en relación a otros Frameworks de PHP.

\end{itemize}